\documentclass{article}

% ================ PACKAGES ====================

\usepackage[a4paper, left=25mm, right=25mm]{geometry}
\usepackage[base]{babel}
\usepackage{lipsum}

% ================ SETTINGS ====================

% no hyphenation
\tolerance=1
\emergencystretch=\maxdimen
\hyphenpenalty=10000
\hbadness=10000

% ================ DOCUMENT ======================

\date{17 June 2023}
\title{CNV Project: Final Report - Group 7}
\author{
    Diogo Neves \\
    \texttt{95554}
    \and
    Filipe Silva \\
    \texttt{95585}
    \and
    João Moniz \\
    \texttt{83480}
}

\begin{document}

\setcounter{page}{0}

\maketitle

\twocolumn

\section{Introduction}

For the final delivery, our objective was to develop an auto scaler and a load
balancer, both in the Java programming language and using the Amazon SDK, that
could do a work similar to the auto scaler and load balancer encountered in the
Amazon Web Services (AWS).

With this objective in mind, we were given a workload (EcoWork@Cloud, in the
\textit{webserver} folder) that created a \textit{webserver} with three scenarios:

\begin{itemize}
    \item Image Compression
    \item Foxes and Rabbits
    \item Insect Wars
\end{itemize}

Our objective was, then, to develop these applications, alongside scripts to
create the images programmatically and deploy them in AWS EC2.

We will then, in this report, show our design decisions for each of the
following modules of our work (in the git repo, each one is a folder located in
the \textit{src} folder):

\begin{itemize}
    \item javassist
    \item lbas
    \item scripts
    \item webserver
\end{itemize}

\section{Design}

\subsection{General}

An overview of the steps taken to deploy this system is as follows:

\begin{enumerate}
    \item Use the scripts to create the images for both the Load Balancer/Auto
        Scaler (\textit{LBAS}) - image name \textit{CNV-LBAS} - and the
        \textit{Webserver} - image name \textit{CNV-Webserver}.
    \item Use the scripts to deploy an EC2 instance with the \textit{CNV-LBAS}
        image, and run it.
    \item This instance will deploy other EC2 instances with the
        \textit{CNV-Webserver} image, according to the load it receives.
    \item Also accordingly to the load, the \textit{LBAS} instance will select
        an instance, using the parameters in the request and the metrics gotten
        from the \textit{webserver} instances (that are running the workload provided
        with a custom Javassist tool to gather and send data to the Load
        Balancer).
    \item It can also choose to launch a Lambda function, if it deems it
        necessary (like when a machine is starting up and cannot receive
        requests yet). It will only use these when necessary, due to their high
        cost relatively to EC2 instances.
\end{enumerate}

\subsection{Javassist}

The Javassist module consists of the \textit{ICount} tool, modified to report
the instruction count per thread (because the workload is multithreaded). It
also used to have the \textit{AmazonDynamoDBConnector} class, which consists of
a custom connector to the Amazon DynamoDB, a key-value store, used to create and
send items to DymanoDB. These will be used to keep certain information about the
program we will run it with. However, it has been moved to the
\textit{Webserver}, and we will explain this decision there.

This module is then ran with the \textit{webserver}, and allows to know how many
instructions were ran for a specific scenario of it, by intercepting each one of
the functions that correspond to a scenario:

\begin{itemize}
    \item \textit{Image Compression} - \textit{process} function, with the
        following arguments: image, target format and compression quality.
    \item \textit{Foxes and Rabbits} - \textit{populate} and
        \textit{runSimulation} functions: the first to get the world that was
        requested (important for the metrics); the second to get the number of
        generations to simuate.
    \item \textit{Insect Wars} - \textit{war} function, with the following
        arguments: size of each army and the maximum number of simulation
        rounds.
\end{itemize}

The number of instructions per thread is saved on a map, and then metrics are
created according to each scenario and sent to DynamoDB in order to be read
later by the Load Balancer, in order to choose an EC2 instance to send the
request to.

In order to keep the number of requests to DynamoDB to a minimum, we only
update the information kept there every \textit{BATCH\_SIZE} requests (default
value is 5, but can be changed in the \textit{ICount} class). The information is
kept in a "cache" (a \textit{Map} in the class for each scenario, alongside a
\textit{Float} for the \textit{InsectWars} scenario: it has no separator between
different types of workloads, such as a world or a type of picture format).

The data sent to DynamoDB is already processed as well (which means that we are
sending to it only the computed statistics and not all the parameters), in order
to reduce computation on the Load Balancer side.

The statistics sent to DynamoDB are the following:

\begin{itemize}
    \item \textit{Image Compression} - the metric chosen was
        \[\frac{I}{width*height*f}\], where \textit{I} is the number of
        instructions, \textit{width} and \textit{height} are the width and
        height of the image and \textit{f} is the compression factor. This
        metric is then sent to DynamoDB for each format: PNG, JPEG and BMP. 
    \item \textit{Foxes and Rabbits} - the metric chosen was \[\frac{I}{G}\], 
        where \textit{I} is the number of instructions and \textit{G} is the
        number of generations. This metric is then sent to DynamoDB for each
        world (the generation doesn't matter much for this workload).
    \item \textit{Insect Wars} - the metric chosen was
        \[\frac{I*r}{max*(sz1+sz2)}\], where \textit{I} is the number of
        instructions, \textit{r} is the ratio between the sizes of the two
        armies, \textit{max} is the maximum number of rounds and \textit{sz1}
        and \textit{sz2} are the sizes of the armies. The ratio is kept in mind
        because it changes somewhat the number of instructions (a bigger
        disparity between armies leads to a quicker battle and less instructions
        ran).
\end{itemize}

It is also relevant to mention that the statistics are updated and not replaced
every 5 requests: it follows an exponential weight, where the new request's
metrics count for 50\% of the new metric, and all the older metrics count for
the other 50\%. This gives us more adaptability to each workload according to
the conditions.

\textbf{Note:} the metrics are sent to DynamoDB just as is said above; however,
they are sent using the \textit{Webserver} and not this package. It used to be
this way but not anymore (again, we will explain this decision on the
\textit{Webserver} section).

We chose to keep this section here because it made
more sense to talk about the metrics collected in the section where we talk
about getting the information used to create these.

So, in conclusion, this module is used to get the number of instructions,
calculate the metrics, and then it sends them to a text file
\textit{/tmp/dynamodb} every \textit{BATCH\_SIZE} requests, where the
\textit{Webserver} will collect and send them.

\subsection{\textit{LBAS}}

\subsubsection{General}

An overview of the \textit{LBAS} operation:
\begin{enumerate}
    \item Creation of a \textit{HashMap} to keep track of the EC2 instances and
        their respective CPU utilizations, alongside two \textit{AtomicInteger}s
        to know how many instances we have (both created and available, as one
        can still be initializing).
    \item Creation of the \textit{AutoScaler} and \textit{LoadBalancer}: while
        the load balancer will work as a webserver, responding to endpoint
        requests, the auto scaler will be running every
        \textit{DELAY\_AUTOSCALER} seconds (default: 10 seconds).
\end{enumerate}

\subsubsection{\textit{Load Balancer}}

Our goal for the load balancer is to distribute the load as evenly as possible
across all instances, in order to keep the number of new instances as low as
possible.

As stated before, the load balancer works as an webserver (much alike the actual
webserver we instrumentalized), and also has a \textit{/test} endpoint in order
to do a sanity check of its operations.

Its process goes as follows:
\begin{enumerate}
    \item Create endpoints with the same names as our workload.
    \item For each request, analyse the parameters given, and calculate an
        estimation of the number of instances that request will take, using data
        gotten from DymanoDB.
        \subitem \textbf{Note:} we keep the metrics in a local cache (using Java
        structures as in the \textit{ICount} class stated before), and only
        every \textit{DYNAMODB\_CACHE} (default: 5) requests do we fetch data
        from DynamoDB. This helps us reduce the latency introduced by going to
        the database every request.
        \subitem \textbf{Note:} if there are no metrics available, the
        estimation will be -1. See below for a better explanation of this
        scenario.
    \item Using this estimation, select the instance to send the request to.
    \item Act as a proxy: send the request and receive the response, then
        forward the response back to the client.
\end{enumerate}

For the instance selection, we need to keep in mind two scenarios:
\begin{enumerate}
    \item There are no metrics available yet: the system just started or is in
        its first requests. In this case, we will choose an instance using
        round-robin: we sort the instances by their CPU average usage (more on
        that in the \textit{Auto Scaler} section), and then iterate through
        these as more requests come by: the instances with least CPU usage will
        be chosen first. Once the map containing the instances and usages is
        updated, so will this list.
    \item There are metrics available: we apply a best-effort algorithm, where
        we sort the instances by the number of instructions (estimated) that it
        has already done, and go from the instance with more instructions ran to
        the one with least ones, trying to "fit" these new instructions in it.

        We associate the instructions ran already with the CPU usage for each
        instance, and then estimate the CPU usage for that request: if the sum
        of the usage for the instance and the estimated usage for the request
        is under 100\%, we assign the request to that instance. That will lead
        to a smaller number of instances being created.
\end{enumerate}

On specific scenarios, we resort to using a Lambda function instead of a node:
this option is more expensive, however it makes up for it for the speed in
startup compared to the time an instance takes to start up.

Especially when the system is under load and more requests are being made while
the LB spawns more nodes, Lambdas will be a good option to support users while
the system is still trying to scale up.

\subsubsection{\textit{Auto Scaler}}

The auto scaler's goal is to respond to higher and lower loads than usual,
adding and removing instances as needed.

The metric we need to keep in mind is the \textbf{average CPU utilization of all
nodes}. Using this, we can define thresholds to scale up and down as needed.
These are defined as follows:

\begin{itemize}
    \item \textit{MAX\_CPU\_USAGE} (default: 80\%)
    \item \textit{MIN\_CPU\_USAGE} (default: 20\%)
\end{itemize}

Therefore, the auto scaler works as follows:

\begin{enumerate}
    \item Iterate over every instance that is running. If it is running and the
        \textit{AMI\_ID} is the one for the image \textit{CNV-Webserver}, we get
        its average CPU usage for the last 60 seconds.
    \item Calculate the average CPU usage of all nodes.
    \item Accordingly scale up or down:
        \subitem If there are no instances, start a new one. However, if one is
        in the \textit{pending} state, do not start a new one just yet.
        \subitem If the average CPU usage is above \textit{MAX\_CPU\_USAGE}, we
        start a new instance.
        \subitem If the average CPU usage is under \textit{MIN\_CPU\_USAGE}, we
        stop the instance with lowest CPU usage. However, it is not stopped
        right there and then: it is marked for termination, and will only be
        terminated after \textit{DELAY\_KILL} (default: 60) seconds. This is
        because we want to make sure that the instance is not needed anymore, it
        might still be running requests.
\end{enumerate}

\subsection{Scripts}

The \textit{scripts} folder contains the scripts to create the images and deploy
the first instance of this system - the \textit{LBAS}.

Here are the scripts alongside a quick explanation for each:

\begin{itemize}
    \item \textit{config.sh} - keeps all the variables needed to contact AWS. Is
        created by the user but there exists a \textit{config.sh.example} that
        acts as a template.
    \item \textit{create-image.sh} - creates the \textit{webserver} and \textit{LBAS}
        images. It is composed of three other scripts:
    \subitem \textit{install-vm.sh} - Installs the needed packages and programs
        in an EC2 instance.
    \subitem \textit{launch-vm.sh} - Launches an EC2 instance. Is used on
        \textit{create-image.sh} but also as a standalone, to launch the
        \textit{LBAS} instance.
    \subitem \textit{test-vm.sh} - Tests that the instance is ready to go, using
        the \textit{/test} endpoint that is found in all programs used in this
        system.
    \item \textit{create-lambda.sh} - creates the lambdas used in this system.
    \item \textit{launch-deployment.sh} - creates all images, lambdas and
        launches the entrypoint to the system (the \textit{LBAS} instance).
\end{itemize}

Each instance (and script) needs to source the script \textit{config.sh}, which
keeps the environment variables needed for the scripts to work, relating to
security groups, key pairs, etc.

\subsection{\textit{Webserver}}

falar do endpoint de test

\section{Things to improve}

falar de load balancing e esperar para apagar instance

\section{Conclusion}

\end{document}