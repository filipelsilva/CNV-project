\documentclass{article}

% ================ PACKAGES ====================

\usepackage[a4paper, left=25mm, right=25mm]{geometry}
\usepackage[base]{babel}
\usepackage{lipsum}

% ================ SETTINGS ====================

% no hyphenation
\tolerance=1
\emergencystretch=\maxdimen
\hyphenpenalty=10000
\hbadness=10000

% ================ DOCUMENT ======================

\date{17 June 2023}
\title{CNV Project: Final Report - Group 7}
\author{
    Diogo Neves \\
    \texttt{95554}
    \and
    Filipe Silva \\
    \texttt{95585}
    \and
    João Moniz \\
    \texttt{83480}
}

\begin{document}

\setcounter{page}{0}

\maketitle

\twocolumn

\section{Introduction}

For the final delivery, our objective was to develop an auto scaler and a load
balancer, both in the Java programming language and using the Amazon SDK, that
could do a work similar to the auto scaler and load balancer encountered in the
Amazon Web Services (AWS).

With this objective in mind, we were given a workload (EcoWork@Cloud, in the
\textit{webserver} folder) that created a webserver with three scenarios:

\begin{itemize}
    \item Image Compression
    \item Foxes and Rabbits
    \item Insect Wars
\end{itemize}

Our objective was, then, to develop these applications, alongside scripts to
create the images programmatically and deploy them in AWS EC2.

We will then, in this report, show our design decisions for each of the
following modules of our work (in the git repo, each one is a folder located in
the \textit{src} folder):

\begin{itemize}
    \item javassist
    \item lbas
    \item scripts
    \item webserver
\end{itemize}

\section{Design}

\subsection{General}

An overview of the steps taken to deploy this system is as follows:

\begin{enumerate}
    \item Use the scripts to create the images for both the Load Balancer/Auto
        Scaler (\textit{LBAS}) - image name \textit{CNV-LBAS} - and the
        Webserver - image name \textit{CNV-Webserver}.
    \item Use the scripts to deploy an EC2 instance with the \textit{CNV-LBAS}
        image, and run it.
    \item This instance will deploy other EC2 instances with the
        \textit{CNV-Webserver} image, according to the load it receives.
    \item Also accordingly to the load, the \textit{LBAS} instance will select
        an instance, using the parameters in the request and the metrics gotten
        from the webserver instances (that are running the workload provided
        with a custom Javassist tool to gather and send data to the Load
        Balancer).
    \item It can also choose to launch a Lambda function, if it deems it
        necessary (like when a machine is starting up and cannot receive
        requests yet). It will only use these when necessary, due to their high
        cost relatively to EC2 instances.
\end{enumerate}

\subsection{Javassist}

The Javassist module consists of two main parts: the \textit{ICount} tool,
modified to report the instruction count per thread (because the workload is
multithreaded); and the \textit{AmazonDynamoDBConnector} class, which consists
of a custom connector to the Amazon DynamoDB, a key-value store, used to create
and send items to DymanoDB. These will be used to keep certain information about
the program we will run it with.

This module is then ran with the webserver, and allows to know how many
instructions were ran for a specific scenario of it, by intercepting each one of
the functions that correspond to a scenario:

\begin{itemize}
    \item \textit{Image Compression} - \textit{process} function, with the
        following arguments: image, target format and compression quality.
    \item \textit{Foxes and Rabbits} - \textit{populate} and
        \textit{runSimulation} functions: the first to get the world that was
        requested (important for the metrics); the second to get the number of
        generations to simuate.
    \item \textit{Insect Wars} - \textit{war} function, with the following
        arguments: size of each army and the maximum number of simulation
        rounds.
\end{itemize}

The number of instructions per thread is saved on a map, and then metrics are
created according to each scenario, and then sent to DynamoDB in order to be
read later by the Load Balancer, in order to choose an EC2 instance to send the
request to.

In order to keep the number of requests to DynamoDB to a minimum, we only
update the information kept there every \textit{BATCH\_SIZE} requests (default
value is 5, but can be changed in the \textit{ICount} class).

The data sent to DynamoDB is already processed as well (which means that we are
sending to it only the computed statistics and not all the parameters), in order
to reduce computation on the Load Balancer side.

The statistics sent to DynamoDB are the following:

\begin{itemize}
    \item \textit{Image Compression} - the metric chosen was
        \[\frac{I}{width*height*f}\], where \textit{I} is the number of
        instructions, \textit{width} and \textit{height} are the width and
        height of the image and \textit{f} is the compression factor. This
        metric is then sent to DynamoDB for each format: PNG, JPEG and BMP. 
    \item \textit{Foxes and Rabbits} - the metric chosen was \[\frac{I}{G}\], 
        where \textit{I} is the number of instructions and \textit{G} is the
        number of generations. This metric is then sent to DynamoDB for each
        world (the generation doesn't matter much for this workload).
    \item \textit{Insect Wars} - the metric chosen was
        \[\frac{I*r}{max*(sz1+sz2)}\], where \textit{I} is the number of
        instructions, \textit{r} is the ratio between the sizes of the two
        armies, \textit{max} is the maximum number of rounds and \textit{sz1}
        and \textit{sz2} are the sizes of the armies. The ratio is kept in mind
        because it changes somewhat the number of instructions (a bigger
        disparity between armies leads to a quicker battle and less instructions
        ran).
\end{itemize}

It is also relevant to mention that the statistics are updated and not replaced
every 5 requests: it follows an exponential weight, where the new request's
metrics count for 50\% of the new metric, and all the older metrics count for
the other 50\%. This gives us more adaptability to each workload according to
the conditions.

\subsection{LBAS}

\subsubsection{General}


\subsubsection{Load Balancer}


\subsubsection{Auto Scaler}


\subsection{Scripts}

The \textit{scripts} folder contains the scripts to create the images and deploy
the first instance of this system - the \textit{LBAS}.

Here are the scripts alongside a quick explanation for each:

% TODO

Each instance (and script) needs to source the script \textit{config.sh}, which
keeps the environment variables needed for the scripts to work, relating to
security groups, key pairs, etc.

\subsection{Webserver}

For the intermediate delivery, we have implemented the following features:
\begin{itemize}
    \item Instrumentation of the EcoWork@Cloud workload using a JavassistAgent:
        in this case we modified the ICount JavassistAgent to be mindful of the
        multithreaded nature of the workload, saving (locally, for now) the
        instruction count of every thread in a map;
    \item Scripts to create the image and deploy/terminate deployment in AWS,
        along with load balancer and auto scaler provided by AWS.
\end{itemize}

For the final delivery, the following goals apply:
\begin{itemize}
    \item Update the (already) modified ICount JavassistAgent to send its
        statistics to DynamoDB (just periodically, to keep performance), so that
        multiple runs can be better distributed across nodes;
    \item Implement a load balancer in Java that uses the information gained
        from ICount (retrieved from DynamoDB) to distribute the load across
        nodes;
    \item Implement an auto scaler in Java that uses the CPU utilizations of the
        various nodes to determine if a node needs creating/destroying.
\end{itemize}

Regarding the load balancer, our goal is to distribute the load as evenly as
possible across all nodes, in order to keep the number of nodes spawned at a
minimum. It can also choose to serve a request using a Lambda instead of a
nodes: this option is more expensive, however it makes up for it for the speed
in startup. Especially when the system is under load and more requests are being
made while the LB spawns more nodes, Lambdas will be a good option to support
users while the system is still trying to scale up.

For the metrics used by the load balancer, we will keep in mind the number of
instructions executed per request, with different granularity based on the
workload that is provided:

\begin{itemize}
    \item \textit{Image Compression} - there are 3 image formats: PNG, JPEG and
        BMP. We will keep a record of the number of instructions per image size,
        per compression factor and format;
    \item \textit{Foxes and Rabbits} - we will keep a record of number of
        instructions per generation, per world (the generation doesn't matter
        much in this case);
    \item \textit{Insect Wars} - we will keep a record of number of instructions
        per round, per number of insects (total) and ratio between the numbers
        of each army.
\end{itemize}

Regarding the auto scaler, the metric we will keep in mind is the average CPU
utilization of each node: if the system detects that more requests are being
made and the CPU utilization is above a certain threshold, it will spawn a new
node. Likewise, if the CPU utilization is below a certain threshold, it will
destroy nodes, in order to keep the cost of running as low as possible. The
algorithm will scale up by creating new nodes, and scale down by marking the
nodes it wants to destroy for termination, however they can only be terminated
once there are no more requests being served by them.

\end{document}