\documentclass{article}

% ================ PACKAGES ====================

\usepackage[a4paper, left=25mm, right=25mm]{geometry}
\usepackage[base]{babel}
\usepackage{lipsum}

% ================ SETTINGS ====================

% no page numbers and the sorts
\pagenumbering{gobble}

% no hyphenation
\tolerance=1
\emergencystretch=\maxdimen
\hyphenpenalty=10000
\hbadness=10000

% ================ DOCUMENT ======================

\date{27 May 2023}
\title{CNV Project Report - Group 7}
\author{
    Diogo Neves \\
    \texttt{95554}
    \and
    Filipe Silva \\
    \texttt{95585}
    \and
    João Moniz \\
    \texttt{83480}
}

\begin{document}

\twocolumn

\maketitle

For the intermediate delivery, we have implemented the following features:
\begin{itemize}
    \item Instrumentation of the EcoWork@Cloud workload using a JavassistAgent:
        in this case we modified the ICount JavassistAgent to be mindful of the
        multithreaded nature of the workload, saving (locally, for now) the
        instruction count of every thread in a map;
    \item Scripts to create the image and deploy/terminate deployment in AWS,
        along with load balancer and auto scaler provided by AWS.
\end{itemize}

For the final delivery, the following goals apply:
\begin{itemize}
    \item Update the (already) modified ICount JavassistAgent to send its
        statistics to DynamoDB (just periodically, to keep performance), so that
        multiple runs can be better distributed across nodes;
    \item Implement a load balancer in Java that uses the information gained
        from ICount (retrieved from DynamoDB) to distribute the load across
        nodes;
    \item Implement an auto scaler in Java that uses the CPU utilizations of the
        various nodes to determine if a node needs creating/destroying.
\end{itemize}

Regarding the load balancer, our goal is to distribute the load as evenly as
possible across all nodes, in order to keep the number of nodes spawned at a
minimum. It can also choose to serve a request using a Lambda instead of a
nodes: this option is more expensive, however it makes up for it for the speed
in startup. Especially when the system is under load and more requests are being
made while the LB spawns more nodes, Lambdas will be a good option to support
users while the system is still trying to scale up.

For the metrics used by the load balancer, we will keep in mind the number of
instructions executed per request, with different granularity based on the
workload that is provided:

\begin{itemize}
    \item \textit{Image Compression} - there are 3 image formats: PNG, JPEG and
        BMP. We will keep a record of the number of instructions per image size,
        per compression factor and format;
    \item \textit{Foxes and Rabbits} - we will keep a record of number of
        instructions per generation, per world (the generation doesn't matter
        much in this case);
    \item \textit{Insect Wars} - we will keep a record of number of instructions
        per round, per number of insects (total) and ratio between the numbers
        of each army.
\end{itemize}

Regarding the auto scaler, the metric we will keep in mind is the average CPU
utilization of each node: if the system detects that more requests are being
made and the CPU utilization is above a certain threshold, it will spawn a new
node. Likewise, if the CPU utilization is below a certain threshold, it will
destroy nodes, in order to keep the cost of running as low as possible. The
algorithm will scale up by creating new nodes, and scale down by marking the
nodes it wants to destroy for termination, however they can only be terminated
once there are no more requests being served by them.

\end{document}